\documentclass{article}
\usepackage{amsmath, amsthm, amsfonts, mathtools}
\begin{document}

\section*{Problem 1}

\begin{proof}
Suppose $S$ is a nonempty subset of $V$. Then clearly $conv(S)$ is nonempty (a necessary condition for $conv(S)$ to be convex). Choose $x, y \in conv(S)$ and $\lambda \in [0, 1]$. Then we have that:
\begin{align*}
\lambda x + (1 - \lambda)y &= \lambda (\gamma_{1}x_1 + \dots + \gamma_{m}x_m) + (1 - \lambda)(\delta_{1}y_1 + \dots + \delta_{n}y_n) \\
&=  \lambda\gamma_{1}x_1 + \dots + \lambda\gamma_{m}x_m + (1 - \lambda)\delta_{1}y_1 + \dots + (1 - \lambda)\delta_{n}y_n \\
\end{align*}
Note that the resulting sum is a convex combination of at most $m + n \in \mathbb{N}$ elements of $S$. Each coefficient is nonnegative (since each $\gamma_i$ and $\delta_i$ is nonnegative, and both $\lambda$ and $1 - \lambda$ are nonnegative). Furthermore, the sum of the coefficients:
$$\lambda\gamma_{1} + \dots + \lambda\gamma_{m} + (1 - \lambda)\delta_{1} + \dots + (1 - \lambda)\delta_{n} = \lambda\sum\limits_{i = 1}^{m}\gamma_i + (1 - \lambda)\sum\limits_{j = 1}^{n}\delta_j = \lambda \cdot 1 + (1 - \lambda) \cdot 1 = 1$$
as required. Thus, $\lambda x + (1 - \lambda)y \in conv(S)$, so $conv(S)$ is convex.
\end{proof}


\section*{Problem 2}

\subsection*{Part (i)}

\begin{proof}
Consider a hyperplane $P$ in $V$ s.t. $P = \{x \in V \mid \langle a, x\rangle = b\}$ with $a \in V$, $a \ne 0$, $b \in \mathbb{R}$. Choose $x, y \in P$ and $\lambda \in [0, 1]$. Then we have that:
\begin{align*}
\langle a, \lambda x + (1 - \lambda)y \rangle &= \lambda\langle a, x \rangle + (1 - \lambda)\langle a, y \rangle \\
&= \lambda b + (1 - \lambda)b \\
&= b \\
\end{align*}
Thus, $\lambda x + (1 - \lambda)y \in P$ by definition. Thus, $P$ is convex.
\end{proof}

\subsection*{Part (ii)}

\begin{proof}
Consider a hyperplane $H$ in $V$ s.t. $H = \{x \in V \mid \langle a, x\rangle \leq b\}$ with $a \in V$, $a \ne 0$, $b \in \mathbb{R}$. Choose $x, y \in H$ and $\lambda \in [0, 1]$. Then we have that:
\begin{align*}
\langle a, \lambda x + (1 - \lambda)y \rangle &= \lambda\langle a, x \rangle + (1 - \lambda)\langle a, y \rangle \\
&\leq \lambda b + (1 - \lambda)b \\
&\leq b \\
\end{align*}
Thus, $\lambda x + (1 - \lambda)y \in H$ by definition. Thus, $H$ is convex.
\end{proof}


\section*{Problem 4}

\subsection*{Part (i)}

\begin{proof}
\begin{align*}
\|x - y\|^2 &= \langle x - y, x - y \rangle \\
&= \langle x - p + p - y, x - p + p - y \rangle \\
&= \langle x - p + p - y, x - p \rangle + \langle x - p + p - y, p - y \rangle \\
&= \langle x - p, x - p + p - y \rangle + \langle p - y, x - p + p - y \rangle \\
&= \langle x - p, x - p \rangle + \langle x - p, p - y \rangle + \langle p - y, x - p \rangle + \langle p - y, p - y \rangle \\
&= \langle x - p, x - p \rangle + \langle p - y, p - y \rangle + 2 \langle x - p, p - y \rangle \\
&= \|x - p\|^2 + \|p - y\|^2 + 2 \langle x - p, p - y \rangle \\
\end{align*}
\end{proof}

\subsection*{Part (ii)}

\begin{proof}
\begin{align*}
\|x - y\|^2 &= \|x - p\|^2 + \|p - y\|^2 + 2 \langle x - p, p - y \rangle \\
&\geq \|x - p\|^2 + \|p - y\|^2 &&\text{by (7.14)} \\
&> \|x - p\|^2 &&\text{since $y \ne p$} \\
\end{align*}
Thus, since $\|x - y\| \geq 0$ and $\|x - p\| \geq 0$, it follows that $\|x - y\| > \|x - p\|$.
\end{proof}

\subsection*{Part (iii)}

\begin{proof}
\begin{align*}
\|x - z\|^2 &= \|x - \lambda y - (1 - \lambda)p\|^2 \\
&= \langle x - \lambda y - (1 - \lambda)p, x - \lambda y - (1 - \lambda)p \rangle \\
&= \langle x - \lambda y - (1 - \lambda)p, x - p\rangle + \lambda\langle x - \lambda y - (1 - \lambda)p, p - y \rangle \\
&=  \langle x - p, x - \lambda y - (1 - \lambda)p\rangle + \lambda\langle p - y, x - \lambda y - (1 - \lambda)p \rangle \\
&= \langle x - p, x - p\rangle + \lambda\langle x - p, p - y \rangle + \lambda\langle p - y, x - p\rangle + \lambda^2\langle p - y, p - y \rangle \\
&= \|x - p\|^2 + 2\lambda\langle x - p, p - y \rangle + \lambda^2\|y - p\|^2 \\
\end{align*}
\end{proof}

\subsection*{Part (iv)}

\begin{proof}
Since $p \in C$ is a projection of x onto $C$, we know that $\|x - z\|^2 \geq \|x - p\|^2$ for $z \in C$ s.t. $z = \lambda y - (1 - \lambda)p$ for some $y \in C$, $\lambda \in [0, 1]$. So, we have that:
\begin{align*}
0 &\leq \|x - z\|^2 - \|x - p\|^2 \\
&= 2\lambda\langle x - p, p - y \rangle + \lambda^2\|y - p\|^2 &&\text{by part (iii)} \\
\end{align*} 
In particular, $0 \leq 2\langle x - p, p - y \rangle + \lambda\|y - p\|^2$ $\forall y \in C, \lambda =0$, i.e. $0 \leq \langle x - p, p - y \rangle$ $\forall y \in C$.
\end{proof}


\section*{Problem 6}

\begin{proof}
Consider the set $S \coloneqq \{x \in \mathbb{R}^n \mid f(x) \leq c\}$. Choose $x, y \in S$ and $\lambda \in [0, 1]$. Then we have that:
\begin{align*}
f(\lambda x + (1 - \lambda)y) &\leq \lambda f(x) + (1 - \lambda)f(y) \\
&\leq \lambda c + (1 - \lambda)c \\
&= c
\end{align*}
Thus, $\lambda x + (1 - \lambda)y \in S$, so $S$ is convex.
\end{proof}


\section*{Problem 7}
Consider $x, y \in C$ and $\gamma \in [0, 1]$. Then we have that:
\begin{proof}
\begin{align*}
f(\gamma x + (1 - \gamma)y) &= \sum\limits_{i = 1}^{k} \lambda_{i}f_{i}(\gamma x + (1 - \gamma) y) \\
&\leq \sum\limits_{i = 1}^{k} \lambda_{i}(\gamma f_{i}(x) + (1 - \gamma)f_{i}(y)) \\
&= \gamma\sum\limits_{i = 1}^{k} \lambda_{i}f_{i}(x) + (1 - \gamma)\sum\limits_{i = 1}^{k} \lambda_{i}f_{i}(y) \\
&= \gamma f(x) + (1 - \gamma)f(y) \\
\end{align*}
Thus, $f$ is convex by definition.
\end{proof}


\section*{Problem 13}

\begin{proof}
Suppose not. Then $\exists a, b \in \mathbb{R}^n$ s.t. $f(a) \ne f(b)$. Say that $f$ is bounded above by some constant $M \in \mathbb{R}$. Consider the set $S \subset \mathbb{R}^{n + 1}$ with $S \coloneqq \{x \in \mathbb{R}^{n + 1} \mid \exists \lambda \in [0, 1] \ s.t. \ \lambda (a, f(a)) + (1 - \lambda)x = (b, f(b))$. This is the set of all points on the line between $(a, f(a)) \in \mathbb{R}^{n + 1}$ and $(b, f(b)) \in \mathbb{R}^{n + 1}$. Since $f(a) \ne f(b)$, we must have that $\exists y \in S$ s.t. the last element of $y$, call it $y_{n + 1}$, is greater than $M$. Define $c \coloneqq (y_1, \dots, y_n)$. Then we must have that $\exists \lambda \in [0, 1] \ s.t. \ \lambda a + (1 - \lambda)c = b$. So, $f(b) = f(\lambda a + (1 - \lambda)c) \leq \lambda f(a) + (1 - \lambda)f(c)$. But we know that $f(b) = \lambda f(a) + (1 - \lambda)y_{n + 1}$. So, we must have $f(c) \geq y_{n + 1} > M$. So, we have reached a contradiction.
\end{proof}


\section*{Problem 20}

\begin{proof}
Choose $x, y \in \mathbb{R}^n$ and $\lambda \in [0, 1]$. Then we have that:
$$f(\lambda x + (1 - \lambda)y) \leq \lambda f(x) + (1 - \lambda) f(y) \qquad -f(\lambda x + (1 - \lambda)y) \leq -\lambda f(x) - (1 - \lambda) f(y)$$
It follows then that $f(\lambda x + (1 - \lambda)y) = \lambda f(x) + (1 - \lambda) f(y)$. Consider the function $g(x) \coloneqq f(x) - f(0)$. We have that $g(x)$ must be linear since, for $x, y \in \mathbb{R}^n$ and $\alpha \in \mathbb{R}$:
\begin{align*}
g(x) &= f(x) - f(0) \\
&= f(\frac{1}{\alpha}(\alpha x) + (1 - \frac{1}{\alpha})(0)) - f(0) \\
&= \frac{1}{\alpha}f(\alpha x) + (1 - \frac{1}{\alpha})f(0) - f(0) \\
&= \frac{1}{\alpha}f(\alpha x) - \frac{1}{\alpha}f(0) \\
\end{align*}
So, we have that $\alpha g(x) = f(\alpha x) - f(0) = g(\alpha x)$. Also, note that:
\begin{align*}
g(x + y) &= g(2(\frac{1}{2}x + \frac{1}{2}y)) \\
&= 2g(\frac{1}{2}x + \frac{1}{2}y) \\
&= 2(f(\frac{1}{2}x + \frac{1}{2}y) - f(0)) \\
&= 2(\frac{1}{2}f(x) + \frac{1}{2}f(y) - f(0)) \\
&= f(x) - f(0) + f(y) - f(0) \\
&= g(x) + g(y) \\
\end{align*}
Thus, since $f(x) = g(x) + f(0)$, we have that $f(x)$ affine by definition.
\end{proof}


\section*{Problem 21}

\begin{proof}
Suppose $x$ is a local minimizer of $f$. Then $\exists \delta > 0$ s.t. $\forall p \in B(x, \delta)$, $f(p) \geq f(x)$. So, $\exists \delta > 0$ s.t. $\forall p \in B(x, \delta)$, $\phi \circ f(p) \geq \phi \circ f(x)$ since $\phi$ is increasing. So, $x$ is a local minimizer of $\phi \circ f$ \\
\\
Suppose $x$ is a local minimizer of $\phi \circ f$. Then $\exists \delta > 0$ s.t. $\forall p \in B(x, \delta)$, $\phi \circ f(p) \geq \phi \circ f(x)$. So, $\exists \delta > 0$ s.t. $\forall p \in B(x, \delta)$, $\phi^{-1} \circ \phi \circ f(p) \geq \phi^{-1} \circ \phi \circ f(x)$ since $\phi$ is strictly increasing so $\phi^{-1}$ is well-defined and increasing. So, $x$ is a local minimizer of $f$.
\end{proof}

\end{document}