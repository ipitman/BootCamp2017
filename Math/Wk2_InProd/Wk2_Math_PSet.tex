\documentclass{article}
\usepackage{amsmath, amsthm, amsfonts, mathtools}
\begin{document}


\section*{Problem 1}

\subsection*{Part (i)}

\begin{proof}
\begin{align*}
\frac{1}{4} (\|x + y\|^2 - \|x - y\|^2) &= \frac{1}{4} (\langle x + y, x + y \rangle - \langle x - y, x - y \rangle) \\
&= \frac{1}{4}(\langle x, x \rangle + \langle x, y \rangle + \langle y, x \rangle + \langle y, y \rangle - \langle x, x \rangle + \langle x, y \rangle + \langle y, x \rangle - \langle y, y \rangle) \\
&=  \frac{1}{4}(4\langle x, y \rangle) \\
&= \langle x, y \rangle \\
\end{align*}
\end{proof}

\subsection*{Part (ii)}

\begin{proof}
\begin{align*}
\frac{1}{2} (\|x + y\|^2 + \|x - y\|^2) &= \frac{1}{2} (\langle x + y, x + y \rangle + \langle x - y, x - y \rangle) \\
&= \frac{1}{2}(\langle x, x \rangle + \langle x, y \rangle + \langle y, x \rangle + \langle y, y \rangle + \langle x, x \rangle - \langle x, y \rangle - \langle y, x \rangle + \langle y, y \rangle) \\
&=  \frac{1}{2}(2\langle x, x \rangle + 2\langle y, y \rangle) \\
&= \langle x, x \rangle + \langle y, y \rangle \\
&= \|x\|^2 + \|y\|^2 \\
\end{align*}
\end{proof}


\section*{Problem 2}

\begin{proof}
Note that:
\begin{align*}
\|x + y\|^2 - \|x - y\|^2  &= \langle x + y, x + y \rangle - \langle x - y, x - y \rangle \\
&= \overline{\langle x, x + y \rangle} + \overline{\langle y, x + y \rangle} - \overline{\langle x, x - y \rangle} + \overline{\langle y, x - y \rangle} \\
&= \overline{\langle x, x \rangle} + \overline{\langle x, y \rangle} + \overline{\langle y, x \rangle} + \overline{\langle y, y \rangle} - \overline{\langle x, x \rangle} + \overline{\langle x, y \rangle} + \overline{\langle y, x \rangle} - \overline{\langle y, y \rangle} \\
&=  2\langle y, x \rangle  + 2\langle x, y \rangle \\
\end{align*}
Also, note that:
\begin{align*}
i\|x - iy\|^2 - i\|x + iy\|^2 &= i\langle x - iy, x - iy \rangle - i\langle x + iy, x + iy \rangle \\
&= \langle x - iy, y \rangle + \langle x - iy, ix \rangle + \langle x + iy, y \rangle - \langle x + iy, ix \rangle \\
&= \overline{\langle y, x \rangle} - \overline{\langle y, iy \rangle} + \overline{\langle ix, x \rangle} - \overline{\langle ix, iy \rangle} + \overline{\langle y, x \rangle} + \overline{\langle y, iy \rangle} - \overline{\langle ix, x \rangle} - \overline{\langle ix, iy \rangle} \\
&= 2\langle x, y \rangle - 2\langle iy, ix \rangle \\
&= 2\langle x, y \rangle - 2\langle y, x \rangle \\
\end{align*}
So, we have that:
\begin{align*}
\frac{1}{4}(\|x + y\|^2 - \|x - y\|^2 + i\|x - iy\|^2 - i\|x + iy\|^2) &= \frac{1}{4}(2\langle y, x \rangle  + 2\langle x, y \rangle + 2\langle x, y \rangle - 2\langle y, x \rangle) \\
&= \frac{1}{4}(4\langle x, y \rangle) \\
&= \langle x, y \rangle \\
\end{align*}
\end{proof}


\section*{Problem 3}

\subsection*{Part (i)}

\begin{align*}
\theta &= \cos^{-1}(\frac{\int_0^1 x^6 dx}{\sqrt{\int_0^1 x^2 dx} \cdot \sqrt{\int_0^1 x^{10} dx}}) \\
&= \cos^{-1}(\frac{\sqrt{33}}{7}) \\
&\approx 0.608
\end{align*}

\subsection*{Part (ii)}

\begin{align*}
\theta &= \cos^{-1}(\frac{\int_0^1 x^6 dx}{\sqrt{\int_0^1 x^4 dx} \cdot \sqrt{\int_0^1 x^{8} dx}}) \\ \\
&= \cos^{-1}(\frac{\sqrt{45}}{7}) \\
&\approx 0.29
\end{align*}


\section*{Problem 8}

\subsection*{Part (i)}

\begin{proof}
Consider $\langle v_i, v_j \rangle$ for $v_i, v_j \in S$, $i \ne j$. Clearly, if $v_i$ is some form of sine and $v_j$ is some form of cosine (or vice versa), then their product is odd and therefore the integral of their product over the interval $[-\pi, \pi]$ is $0$, so $\langle v_i, v_j \rangle = 0$. If $v_i = cos(t)$ and $v_j = cos(2t)$ (or vice versa), we have that:
$$\langle v_i, v_j \rangle = \frac{1}{\pi}\int_{-\pi}^\pi cos(t) cos(2t) dt = 0$$
If $v_i = sin(t)$ and $v_j = sin(2t)$ (or vice versa), we have that:
$$\langle v_i, v_j \rangle = \frac{1}{\pi}\int_{-\pi}^\pi sin(t) sin(2t) dt = 0$$
Consider $\langle v_i, v_j \rangle$ for $v_i, v_j \in S$, $i = j$. Note that:
$$\int_{-\pi}^\pi cos^2(t) dt = \int_{-\pi}^\pi sin^2(t) dt = \int_{-\pi}^\pi cos^2(2t) dt = \int_{-\pi}^\pi sin^2(2t) dt = \pi$$
Thus, by our definition of the inner product over V, $\langle v_i, v_j \rangle = 1$. Thus, S is an orthonormal set.
\end{proof}

\subsection*{Part (ii)}

$$\|t\| = \sqrt{\frac{1}{\pi}\int_{-\pi}^\pi t^2 dt} = \sqrt{\frac{2}{3}} \pi$$

\subsection*{Part (iii)}

\begin{align*}
proj_{X}(cos(3t)) &= \langle cos(3t), cos(t) \rangle \cdot cos(t) + \langle cos(3t), sin(t) \rangle \cdot sin(t)\ +\\
			&\quad + \langle cos(3t), cos(2t) \rangle \cdot cos(2t) + \langle cos(3t), sin(2t) \rangle \cdot sin(2t) \\
&= 0 \cdot cos(t) + 0 \cdot sin(t) + 0 \cdot cos(2t) + 0 \cdot sin(2t)\\
&= 0
\end{align*}

\subsection*{Part (iv)}

\begin{align*}
proj_{X}(t) &= \langle t, cos(t) \rangle \cdot cos(t) + \langle t, sin(t) \rangle \cdot sin(t) + \langle t, cos(2t) \rangle \cdot cos(2t) + \langle t, sin(2t) \rangle \cdot sin(2t) \\
&= 0 + 2sin(t) + 0 - sin(2t) \\
&= 2sin(t) - sin(2t)
\end{align*}

\section*{Problem 9}

\begin{proof}
We define a rotation $r: \mathbb{R}^2 \rightarrow \mathbb{R}^2$ as:
$$r\bigg(\theta, \begin{bmatrix} x \\ y \end{bmatrix} \bigg) = \begin{bmatrix} x \cdot cos(\theta) - y \cdot sin(\theta) \\ x \cdot sin(\theta) + y \cdot cos(\theta) \end{bmatrix}$$
Consider two vectors $\begin{bmatrix} a \\ b \end{bmatrix}$ and $\begin{bmatrix} c \\ d \end{bmatrix}$ in $\mathbb{R}^2$. Then we have that:
$$\left\langle \begin{bmatrix} a \\ b \end{bmatrix}, \begin{bmatrix} c \\ d \end{bmatrix} \right\rangle = ac + bd$$
Rotating, we get:
\begin{align*}
&\left\langle \begin{bmatrix} a \cdot cos(\theta) - b \cdot sin(\theta) \\ a \cdot sin(\theta) + b \cdot cos(\theta) \end{bmatrix}, \begin{bmatrix} c \cdot cos(\theta) - d \cdot sin(\theta) \\ c \cdot sin(\theta) + d \cdot cos(\theta) \end{bmatrix} \right\rangle = \\
&= (a \cdot cos(\theta) - b \cdot sin(\theta)) \cdot (c \cdot cos(\theta) - d \cdot sin(\theta)) + (a \cdot sin(\theta) + b \cdot cos(\theta)) \cdot (c \cdot sin(\theta) + d \cdot cos(\theta)) \\
&= ac \cdot cos^2(\theta) - ad \cdot sin(\theta) cos(\theta) - bc \cdot sin(\theta) cos(\theta) + bd \cdot sin^2(\theta) \ + \\
&\quad \ + ac \cdot sin^2(\theta) + ad \cdot sin(\theta)cos(\theta) + bc \cdot sin(\theta)cos(\theta) + bd \cdot cos^2(\theta) \\
&= ac \cdot (cos^2(\theta) + sin^2(\theta)) + bd \cdot (cos^2(\theta) + sin^2(\theta)) \\
&= ac + bd
\end{align*}
So, a rotation in $\mathbb{R}^2$ is an orthonormal transformation (with respect to the usual inner product).
\end{proof}

\section*{Problem 10}

\subsection*{Part (i)}

\begin{proof}
Suppose $Q \in M_n(\mathbb{F})$ is an orthonormal matrix. Then we must have $\langle Qu, Qv \rangle = \langle u, v \rangle$ $\forall u, v \in \mathbb{F}^n$. By the definition of the standard inner product:
\begin{align*}
(Qu)^{H}Qv &= u^{H}v \\
u^{H}Q^{H}Qv &= u^{H}v \\
Q^{H}Q &= I \\
\end{align*}
Thus, by the uniqueness of inverses, $QQ^{H} = Q^{H}Q = I$ if $Q \in M_{n}(\mathbb{F})$ is an orthonormal matrix.
Suppose $QQ^{H} = Q^{H}Q = I$. Then for any $u, v \in \mathbb{F}^n$,
\begin{align*}
Q^{H}Q &= I \\
Q^{H}Qv &= v \\
u^{H}Q^{H}Qv &= u^{H}v \\
(Qu)^{H}Qv &= u^{H}v \\
\langle Qu, Qv \rangle &= \langle u, v \rangle \\
\end{align*}
So, $\langle Qu, Qv \rangle = \langle u, v \rangle$ $\forall u, v \in \mathbb{F}^n$, i.e. $Q \in M_n(\mathbb{F})$ is an orthonormal matrix.
\end{proof}

\subsection*{Part (ii)}

\begin{proof}
Consider $x \in \mathbb{F}^n$. Then, we have that:
\begin{align*}
\|Qx\| &= \sqrt{\langle Qx, Qx \rangle} \\
&= \sqrt{\langle x, x \rangle} &&\text{Since $Q \in M_n(\mathbb{F})$ is an orthonormal matrix} \\
&= \|x\|
\end{align*}
\end{proof}

\subsection*{Part (iii)}

\begin{proof}
From part (i), we know that if $Q \in M_n(\mathbb{F})$ is an orthonormal matrix, then $Q^{-1} = Q^H$. Also, we know that if $QQ^{H} = Q^{H}Q = I$, then $Q$ is an orthonormal matrix. Also, note that $Q^{H}(Q^{H})^{H} = Q^{H}Q$ and $(Q^{H})^{H}Q^{H} = QQ^{H}$. So, we have that:
$$Q^{H}(Q^{H})^{H} = (Q^{H})^{H}Q^{H} = QQ^{H} = Q^{H}Q  = I$$
So $Q^{-1} = Q^{H}$ is an orthonormal matrix if $Q \in M_n(\mathbb{F})$ is an orthonormal matrix.
\end{proof}

\subsection*{Part (iv)}

\begin{proof}
Suppose an orthonormal matrix $Q \in M_n(\mathbb{F})$ has column vectors $\{v_1, v_2, \dots, v_n\}$. Then we have that:
$$Q = \begin{bmatrix} v_1 & v_2 & \dots & v_n \end{bmatrix} \qquad Q^H = \begin{bmatrix} v_1^H \\ v_2^H \\ \vdots \\ v_n^H \end{bmatrix}$$
Since Q is orthonormal, we know that:
\begin{align*}
Q^{H}Q &= \begin{bmatrix} v_1^H \\ v_2^H \\ \vdots \\ v_n^H \end{bmatrix} \cdot   \begin{bmatrix} v_1 & v_2 & \dots & v_n \end{bmatrix} \\
&=  \begin{bmatrix} v_1^{H}v_1 & v_1^{H}v_2 & \dots & v_1^{H}v_n \\
v_2^{H}v_1 & v_2^{H}v_2 & \dots & v_2^{H}v_n \\
\vdots & \vdots & \ddots & \vdots \\
v_n^{H}v_1 & v_n^{H}v_2 & \dots & v_n^{H}v_n \end{bmatrix} \\
&= \begin{bmatrix} 1 & 0 & \dots & 0 \\
0 & 1 & \dots & 0 \\
\vdots & \vdots & \ddots & \vdots \\
0 & 0 & \dots & 1 \end{bmatrix} \\
&= I
\end{align*}
So, we have that the column vectors of an orthonormal matrix form an orthonormal set, since $\langle v_i, v_j \rangle = v_i^{H}v_j = 0$ if $i \ne j$, and $\langle v_i, v_j \rangle = v_i^{H}v_j = 1$ if $i = j$.
\end{proof}

\subsection*{Part (v)}

\begin{proof}
For $Q \in M_n(\mathbb{F})$ an orthonormal matrix, we have that:
$$1 = det(I) = det(Q^{H}Q) = det(Q^{H})det(Q) = (det(Q))^2$$
So, $\mid det(Q) \mid = 1$.  The converse is not true; consider $P = \begin{bmatrix} 2 & 0 \\ 0 & \frac{1}{2} \end{bmatrix}$
\end{proof}

\subsection*{Part (vi)}

\begin{proof}
Suppose $Q_1, Q_2 \in M_n(\mathbb{F})$ are orthonormal matrices. Consider their product, $Q_{1}Q_{2}$. We have that:
$$Q_{1}Q_{2}(Q_{1}Q_{2})^H = Q_{1}Q_{2}Q_{2}^{H}Q_{1}^{H} = Q_{1}IQ_{1}^{H} = I$$
Thus, by part (i), $Q_{1}Q_{2}$ is an orthonormal matrix.
\end{proof}

\section*{Problem 11}
If a collection $\{v_1, v_2, \dots, v_n\}$ of vectors is linearly dependent, then $\exists i \in \mathbb{N}$ s.t. $1 \leq i \leq n$ and $v_i \in span(\{v_1, v_2, \dots, v_{i - 1}\})$. WLOG, say that $\{v_1, v_2, \dots, v_{i - 1}\}$ are linearly independent. Note that we define:
$$q_i = \frac{x_i - proj_{span(\{q_1, \dots, q_{i - 1}\})}(x_i)}{\|x_i - proj_{span(\{q_1, \dots, q_{i - 1}\})}(x_i)\|}$$
Note that since $\{v_1, v_2, \dots, v_{i - 1}\}$ are linearly independent, then $span(\{q_1, \dots, q_{i - 1}\}) =  span(\{v_1, v_2, \dots, v_{i - 1}\})$ by construction. So, we get that:
\begin{align*}
proj_{span(\{q_1, \dots, q_{i - 1}\})}(x_i) &= proj_{span(\{v_1, v_2, \dots, v_{i - 1}\})}(x_i) \\
&= x_i &&\text{since $x_i \in span(\{v_1, v_2, \dots, v_{i - 1}\})$} \\
\end{align*}
So, the denominator in the expression for $q_i$ is 0, so $q_i$ is undefined, and thus every $q_j$ with $j \geq i$ is undefined.


\section*{Problem 16}

\subsection*{Part (i)}

\begin{proof}
Suppose we have factored a matrix $A \in M_{m \times n}$ into a product $A = QR$, where $Q$ is an $m \times m$ orthonormal matrix and $R$ is an $m \times n$ upper-triangular matrix. Consider the $m \times m$ matrix:
$$D \coloneqq \begin{bmatrix} -1 & 0 & \dots & 0 \\ 0 & -1 & \dots & 0 \\ \vdots & \vdots & \ddots & \vdots \\ 0 & 0 & \dots & -1 \end{bmatrix}$$
Then the matrix $QD \ne Q$ is still an orthonormal matrix, and the matrix $DR \ne R$ is still an upper-triangular matrix, and:
$$(QD)(DR)  = QR = A$$
Thus, the QR decomposition is not unique.
\end{proof}

\subsection*{Part (ii)}

\begin{proof}
Suppose $A = Q_{1}R_{1} = Q_{2}R_{2}$ with $Q_1, Q_2$ orthonormal matrices and $R_1, R_2$ upper-triangular matrices. Then we have that:
\begin{align*}
R_1^{H}R_1 &= R_1^{H}(Q_1^{H}Q_1)R_1 \\
&= A^{H}A \\
&= R_2^{H}(Q_2^{H}Q_2)R_2 \\
&= R_2^{H}R_2 \\
\end{align*}
So, we have that:
$$(R_2^{H})^{-1}R_1^{H} = R_{2}(R_1)^{-1}$$
Note that the left hand side of the above equation must be a lower-triangular matrix, since the inverse of a lower-triangular matrix is lower-triangular and the product of two lower-triangular matrices is lower-triangular. Also, the right hand side of the equation must be an upper-triangular matrix, since the inverse of an upper-triangular matrix is upper-triangular and the product of two upper-triangular matrices is upper-triangular. Thus, we must have that both sides of the equation are in fact diagonal matrices. \\
\\
Note that the notion of "positive" is not well-defined on $\mathbb{C} \backslash \mathbb{R}$, so we must assert that the diagonal elements of $R_1$ and $R_2$ have no imaginary component in order for them to be positive. Say that the diagonal elements of $R_1$ are $\alpha_i > 0$ for $1 \leq i \leq n$ and the diagonal elements of $R_2$ are $\beta_i > 0$ for $1 \leq i \leq n$. Note then that the diagonal elements of $(R_1)^{-1}$ are $1 / \alpha_i > 0$ for $1 \leq i \leq n$, and that the diagonal elements of $(R_2^{H})^{-1}$ are $1 / \beta_i > 0$ for $1 \leq i \leq n$. So, since the diagonal of the product of two triangular matrices of the same kind (i.e. both upper or both lower) is the element-wise product of their diagonals, we have that:
$$\frac{\alpha_i}{\beta_i} = \frac{\beta_i}{\alpha_i} \ \text{for $1 \leq i \leq n$}$$
Since $\alpha_i > 0$ and $\beta_i > 0$ for $1 \leq i \leq n$, we have that $\alpha_i = \beta_i$ for $1 \leq i \leq n$. Consequently,
$$(R_2^{H})^{-1}R_1^{H} = R_{2}(R_1)^{-1} = I$$
So, by the uniqueness of inverses, we must have that $R_1 = R_2$. Since $Q_{1}R_{1} = Q_{2}R_{2}$, it follows that $Q_{1} = Q_{2}$. So, There is a unique $QR$ decomposition of $A$ s.t. $R$ has only positive diagonal elements.
\end{proof}


\section*{Problem 17}

\begin{proof}
\begin{align*}
A^{H}Ax &= A^{H}b \\
(\widehat{Q}\widehat{R})^{H}\widehat{Q}\widehat{R}x &= (\widehat{Q}\widehat{R})^{H}b \\
\widehat{R}^{H}\widehat{Q}^{H}\widehat{Q}\widehat{R}x &= \widehat{R}^{H}\widehat{Q}^{H}b \\
\widehat{R}^{H}I\widehat{R}x &= \widehat{R}^{H}\widehat{Q}^{H}b \\
\widehat{R}x &= \widehat{Q}^{H}b \\
\end{align*}
\end{proof}


\section*{Problem 23}

\begin{proof}
Note that:
\begin{align*}
\|x\| &= \|x - y + y\| \\
&\leq \|x - y\| + \|y\| \\
\end{align*}
So, we have that $\|x\| - \|y\| \leq \|x - y\|$ Exchanging $x$ and $y$, we get that $\|y\| - \|x\| \leq \|x - y\|$ since $\|y - x\| = \|x - y\|$. Thus, since $\lvert\|x\| - \|y\|\rvert \in \{\|x\| - \|y\|, \|y\| - \|x\|\}$, we have that:
$$\lvert\|x\| - \|y\|\rvert \leq \|x - y\| \ \forall \ x, y \in V$$
\end{proof}


\section*{Problem 24}

\subsection*{Part (i)}

\begin{proof}
Consider the norm:
$$\|f\|_{L^{1}} = \int_a^b \lvert f(t) \rvert dt$$
Then, we have that $\|f\|_{L^{1}} \geq 0$ and $\|f\|_{L^{1}} = 0$ iff $f$ is the zero function. This is a direct result of the fact that $\mid \cdot \mid$ has this property on $\mathbb{F}$. We also have that, for some constant $c \in \mathbb{F}$:
\begin{align*}
\|cf\|_{L^{1}} &=  \int_a^b \lvert cf(t) \rvert dt \\
&=  \int_a^b \lvert c \rvert \cdot \lvert f(t) \rvert dt \\
&=  \lvert c \rvert \int_a^b \lvert f(t) \rvert dt \\
&=  \lvert c \rvert \cdot \|f\|_{L^{1}}
\end{align*}
Finally, we have that:
\begin{align*}
\|f\|_{L^{1}} + \|g\|_{L^{1}} &=  \int_a^b \lvert f(t) \rvert dt +  \int_a^b \lvert g(t) \rvert dt \\
&=  \int_a^b \lvert f(t) \rvert + \lvert g(t) \rvert dt \\
&\geq  \int_a^b \lvert f(t) + g(t) \rvert dt \\
&= \|f + g\|_{L^{1}} \\
\end{align*}
\end{proof}

\begin{proof}
Consider the norm:
$$\|f\|_{L^{2}} = \left(\int_a^b \lvert f(t) \rvert^2 dt\right)^{1 / 2}$$
Then, we have that $\|f\|_{L^{1}} \geq 0$ and $\|f\|_{L^{1}} = 0$ iff $f$ is the zero function. This is a direct result of the fact that $\mid \cdot \mid$ has this property on $\mathbb{F}$. We also have that, for some constant $c \in \mathbb{F}$:
\begin{align*}
\|cf\|_{L^{2}} &=  \left(\int_a^b \lvert cf(t) \rvert^2 dt\right)^{1 / 2}\\
&= \left(\int_a^b \lvert c \rvert^2\lvert f(t) \rvert^2 dt\right)^{1 / 2} \\
&=  \lvert c \rvert \cdot \left(\int_a^b \lvert f(t) \rvert^2 dt\right)^{1 / 2} \\
&=  \lvert c \rvert \cdot \|f\|_{L^{2}}
\end{align*}
Finally, we have that:
\begin{align*}
(\|f + g\|_{L^{2}})^2 &= \int_a^b \lvert f(t) + g(t) \rvert^2 dt \\
&\leq \int_a^b (\lvert f(t) \rvert + \lvert g(t) \rvert) \cdot \lvert f(t) + g(t) \rvert dt \\
&= \int_a^b \lvert f(t) \rvert \cdot \lvert f(t) + g(t) \rvert dt + \int_a^b \lvert g(t) \rvert \cdot \lvert f(t) + g(t) \rvert dt \\
&\leq \left( \left(\int_a^b \lvert f(t) \rvert^2 dt\right)^{1 / 2} + \left(\int_a^b \lvert g(t) \rvert^2 dt\right)^{1 / 2} \right) \cdot \left(\int_a^b \lvert f(t) + g(t) \rvert^2 dt\right)^{1 / 2} \text{by H\"older's inequality} \\
&= (\|f\|_{L^{2}} + \|g\|_{L^{2}}) \cdot \|f + g\|_{L^{2}}
\end{align*}
Thus, $\|f + g\|_{L^{2}} \leq \|f\|_{L^{2}} + \|g\|_{L^{2}}$
\end{proof}

\begin{proof}
Consider the norm:
$$\|f\|_{L^{\infty}} = sup_{x \in [a, b]}\lvert f(x) \rvert$$
Then, we have that $\|f\|_{L^{\infty}} \geq 0$ and $\|f\|_{L^{\infty}} = 0$ iff $f$ is the zero function. This is a direct result of the fact that $\mid \cdot \mid$ has this property on $\mathbb{F}$. We also have that, for some constant $c \in \mathbb{F}$:
\begin{align*}
\|cf\|_{L^{\infty}} &=  sup_{x \in [a, b]}\lvert cf(x) \rvert \\
&= sup_{x \in [a, b]}\lvert c \rvert \cdot \lvert f(x) \rvert \\
&= \lvert c \rvert \cdot sup_{x \in [a, b]}\lvert f(x) \rvert \\
&= \lvert c \rvert \cdot \|f\|_{L^{\infty}}
\end{align*}
Finally, we have that:
\begin{align*}
\|f\|_{L^{\infty}} + \|g\|_{L^{\infty}} &=  sup_{x \in [a, b]}\lvert f(x) \rvert + sup_{x \in [a, b]}\lvert g(x) \rvert \\
&\geq sup_{x \in [a, b]}(\lvert f(x) \rvert + \lvert g(x) \rvert) \\
&\geq sup_{x \in [a, b]}\lvert f(x) + g(x) \rvert \\
&= \|f + g\|_{L^{\infty}} \\
\end{align*}
\end{proof}


\section*{Problem 26}

\subsection*{Part (i)}

\begin{proof}
$\| \cdot \|_a \sim \| \cdot \|_a$ by choosing $m = M = 1$ \\
\\
Suppose $\| \cdot \|_a \sim \| \cdot \|_b$. Then $\exists$ constants $0 < m \leq M$ such that:
$$m\|x\|_a \leq \|x_b\| \leq M\|x\|_{a}, \quad \forall x \in X$$
So, $\exists$ constants $0 < \frac{1}{M} \leq \frac{1}{m}$ such that:
$$\frac{1}{M}\|x\|_b \leq \|x_a\| \leq \frac{1}{m}\|x\|_{b}, \quad \forall x \in X$$
i.e. $\| \cdot \|_b \sim \| \cdot \|_a$ \\
\\
Suppose $\| \cdot \|_a \sim \| \cdot \|_b$ and $\| \cdot \|_b \sim \| \cdot \|_c$. Then $\exists$ constants $0 < m \leq M$ and  $0 < n \leq N$ such that:
$$m\|x\|_a \leq \|x_b\| \leq M\|x\|_{a}, \quad \forall x \in X$$
$$n\|x\|_b \leq \|x_c\| \leq N\|x\|_{b}, \quad \forall x \in X$$
So, $\exists$ constants $0 < nm \leq NM$ such that:
$$nm\|x\|_a \leq \|x_c\| \leq NM\|x\|_{a}, \quad \forall x \in X$$
i.e. $\| \cdot \|_a \sim \| \cdot \|_c$ \\
Thus, $\sim$ is an equivalence relation. \\
\\
Define the following $p$-norms for a vector $x \in \mathbb{F}^n$ with components $\{x_1, x_2, \dots, x_n\}$:
$$\|x\|_1 = \sum\limits_{j = 1}^n \lvert x_j \rvert$$
$$\|x\|_2 = \left( \sum\limits_{j = 1}^n \lvert x_j \rvert ^ 2 \right)^{1 / 2}$$
$$\|x\|_{\infty} = sup\{\lvert x_1 \rvert, \lvert x_2 \rvert, \dots, \lvert x_n \rvert \}$$
Then, by the triangle inequality on the 2-norm, we have that:
$$\|x\|_2 = \left( \sum\limits_{j = 1}^n \lvert x_j \rvert ^ 2\right)^{1 / 2} \leq \sum\limits_{j = 1}^n (\lvert x_j \rvert^2)^{1 / 2}  = \sum\limits_{j = 1}^n \lvert x_j \rvert = \|x\|_1$$
Also, by H\"older's inequality, we have that:
$$\|x\|_1 =  \sum\limits_{j = 1}^n \lvert x_j \rvert \leq \left( \sum\limits_{j = 1}^n \lvert 1 \rvert ^ 2\right)^{1 / 2} \cdot \left( \sum\limits_{j = 1}^n \lvert x_j \rvert ^ 2\right)^{1 / 2} = \sqrt{n}\|x\|_2$$
It follows that $\|x\|_2 \leq \|x\|_1 \leq \sqrt{n}\|x\|_2$

\subsection*{Part (ii)}

Since $n$ finite by assumption, we have that:
\begin{align*}
\|x\|_{\infty} &= sup\{\lvert x_1 \rvert, \lvert x_2 \rvert, \dots, \lvert x_n \rvert \} \\
&= max\{\lvert x_1 \rvert, \lvert x_2 \rvert, \dots, \lvert x_n \rvert \} \\
&\coloneqq \lvert x_i \rvert \quad \text{for some $i \in [1, n]$} \\
\end{align*}
It follows that:
\begin{align*}
\|x\|_{\infty} &= \lvert x_i \rvert \\
&= (\lvert x_i \rvert^2)^{1/2} \\
&\leq \left(\sum\limits_{j = 1}^{i - 1} \lvert x_j \rvert^2 + \lvert x_i \rvert^2 + \sum\limits_{k = i + 1}^{n} \lvert x_k \rvert^2 \right)^{1 / 2} \\
&= \left(\sum\limits_{j = 1}^{n} \lvert x_j \rvert^2\right)^{1 / 2} \\
&= \|x\|_{2} \\
\end{align*}
Finally, we have that:
$$\|x\|_{2} = \left(\sum\limits_{j = 1}^{n} \lvert x_j \rvert^2\right)^{1 / 2} \leq \left(\sum\limits_{j = 1}^{n} \lvert x_i \rvert^2\right)^{1 / 2} = (n\lvert x_i \rvert^2)^{1 / 2} = \sqrt{n}\lvert x_i \rvert = \sqrt{n}\|x\|_{\infty}$$
It follows that $\|x\|_{\infty} \leq \|x\|_{2} \leq \sqrt{n}\|x\|_{\infty}$ \\
\\
So, by part (i), we have that $\| \cdot \|_1 \sim \| \cdot \|_2 \sim \| \cdot \|_{\infty}$
\end{proof}

\section*{Problem 28}

\subsection*{Part (i)}
\begin{proof}
By problem 26, we have that:
$$\frac{1}{\sqrt{n}}\|A\|_2 = \sup_{x \ne 0}\frac{\|Ax\|_2}{\sqrt{n}\|x\|_2} \leq \sup_{x \ne 0}\frac{\|Ax\|_2}{\|x\|_1} \leq \sup_{x \ne 0}\frac{\|Ax\|_1}{\|x\|_1} = \|A\|_1$$
Also, we have that:
$$\sqrt{n}\|A\|_2 = \sup_{x \ne 0}\frac{\sqrt{n}\|Ax\|_2}{\|x\|_2} \geq \sup_{x \ne 0}\frac{\|Ax\|_1}{\|x\|_2} \geq \sup_{x \ne 0}\frac{\|Ax\|_1}{\|x\|_1} = \|A\|_1$$
It follows that $\frac{1}{\sqrt{n}}\|A\|_2 \leq \|A\|_1 \leq \sqrt{n}\|A\|_2$
\end{proof}

\subsection*{Part (ii)}

\begin{proof}
Replacing $\|A\|_1$ with $ \|A\|_2$ and  $\|A\|_2$ with $ \|A\|_{\infty}$, we have that $\frac{1}{\sqrt{n}}\|A\|_{\infty} \leq \|A\|_2 \leq \sqrt{n}\|A\|_{\infty}$ by part (i) and the results of problem 26. Thus, we have that the operator $p$-norms are topologically equivalent.
\end{proof}


\section*{Problem 29}

\begin{proof}
Note that by defining $\| \cdot \|_2$ on $\mathbb{F}^n$, we have implicitly defined an inner product $\langle \ \cdot \ \rangle$ on $\mathbb{F}^n$ s.t. for vectors $x, y \in \mathbb{F}^n$ with components $\{x_1, x_2, \dots, x_n\}$ and $\{y_1, y_2, \dots, y_n\}$, we have that:
$$\langle x, y \rangle = \sum\limits_{i = 1}^n x_{i}y_{i}$$
Thus, for $x \in \mathbb{F}^n$, we have that $\|x\|_2 = \sqrt{\langle x, x \rangle}$ \\
$$\|Q\| = \sup_{x \ne 0} \frac{\|Qx\|_2}{\|x\|_2} = \sup_{x \ne 0} \frac{\sqrt{\langle Qx, Qx \rangle}}{\sqrt{\langle x, x \rangle}} =\sup_{x \ne 0} \frac{\sqrt{\langle x, x \rangle}}{\sqrt{\langle x, x \rangle}} = 1$$
Note that:
$$\|R_x\| = \sup_{A \ne 0} \frac{\|Ax\|_2}{\|A\|} =  \sup_{A \ne 0} \frac{\|Ax\|_2}{\left(\sup_{x \ne 0} \frac{\|Ax\|_2}{\|x\|_2}\right)} \leq \sup_{A \ne 0} \frac{\|Ax\|_2}{\frac{\|Ax\|_2}{\|x\|_2}} = \sup_{A \ne 0} \|x\|_2 = \|x\|_2$$
Don't know how to prove equality.
\end{proof}

\section*{Problem 30}

\begin{proof}
Clearly, $\|A\|_S \geq 0$ $\forall A \in M_n(\mathbb{F})$ since $SAS^{-1} \in M_n(\mathbb{F})$ and $\|A\| \geq 0$ $\forall A \in M_n(\mathbb{F})$. Suppose $\|A\|_S = 0$. Then we have that:
\begin{align*}
\|SAS^{-1}\| &= 0 \\
SAS^{-1} &= 0 \\
SAS^{-1}S &= 0S \\
SA &= 0 \\
S^{-1}SA &= S^{-1}0 \\
A &= 0 \\
\end{align*}
Thus, if $\|A\|_S = 0$, then $A = 0$.\\
\\
Consider $\|cA\|_S$ for some constant $c \in \mathbb{F}$. Then,
\begin{align*}
\|cA\|_S &= \|S(cA)S^{-1}\| \\
&= \|cSAS^{-1}\| \\
&= \lvert c \rvert \cdot \|SAS^{-1}\| \\
&= \lvert c \rvert \cdot \|A\|_S \\
\end{align*}
Finally, consider $\|A + B\|_S$ for $A, B \in M_n(\mathbb{F})$:
\begin{align*}
\|A + B\|_S &= \|S(A + B)S^{-1}\| \\
&= \|(SA + SB)S^{-1}\| \\
&= \|SAS^{-1} + SBS^{-1}\| \\
&\leq \|SAS^{-1}\| + \|SBS^{-1}\| \\
&= \|A\|_S + \|B\|_S
\end{align*}
Thus, $\| \cdot \|_S$ is a matrix norm on $M_n(\mathbb{F})$
\end{proof}


\section*{Problem 37}
\begin{proof}
We want to find $d, e, f \in \mathbb{R}$ s.t. $\forall a, b, c \in \mathbb{R}$:
$$\int_0^1 (dx^2 + ex + f)(ax^2 + bx + c) dx = 2a + b$$
By evaluating the above integral and solving a system of 3 equations, 3 variables, we obtain:
$$d = 180, e = -168, f = 24$$
Thus, we have the required $q \in V$, namely $q = 180x^2 - 168x + 24$
\end{proof}


\section*{Problem 38}

\begin{proof}
The matrix representation of D with respect to the power basis $[1, x, x^2]$ of $\mathbb{F}[x; 2]$ is:
$$\begin{bmatrix}
0 & 1 & 0 \\
0 & 0 & 2 \\
0 & 0 & 0 \\
\end{bmatrix}$$
The matrix representation of the adjoint of D is:
$$\begin{bmatrix}
0 & -1 & 0 \\
0 & 0 & -2 \\
0 & 0 & 0 \\
\end{bmatrix}$$
I can't figure out the boundary conditions (how to get rid of $f(1)g(1) - f(0)g(0)$).
\end{proof}

\section*{Problem 39}

\subsection*{Part (i)}

\begin{proof}
\begin{align*}
\langle (S + T)^{*}(w), v \rangle_V &= \langle w, (S + T)(v) \rangle_W \\
&= \langle w, S(v) \rangle_W + \langle w, T(v) \rangle_W \\
&= \langle S^{*}(w), v \rangle_V + \langle T^{*}(w), v \rangle_V \\
&= \overline{\langle v , S^{*}(w) \rangle}_V + \overline{\langle v , T^{*}(w) \rangle}_V \\
&= \overline{\langle v , S^{*}(w) + T^{*}(w) \rangle_V} \\
&= \langle S^{*}(w) + T^{*}(w), v \rangle_V
\end{align*}
By uniqueness of the adjoint, then, we must have $(S + T)^{*} = S^{*} + T^{*}$.
\begin{align*}
\langle (\alpha T)^{*}(w), v \rangle_V &= \langle w, (\alpha T)(v) \rangle_W \\
&= \alpha \langle w, T(v) \rangle_W \\
&= \alpha \langle T^{*}(w), v \rangle_V \\
&= \langle \overline{\alpha} T^{*}(w), v \rangle_V
\end{align*}
By uniqueness of the adjoint, then, we must have $(\alpha T)^{*} = \overline{\alpha} T^{*}$
\end{proof}

\subsection*{Part (ii)}
\begin{proof}
$$\langle w, (S^{*})^{*}(v) \rangle_W = \langle S^{*}(w), v \rangle_V = \langle w, S(v) \rangle_W$$
\end{proof}

\subsection*{Part (iii)}

\begin{proof}
\begin{align*}
\langle (ST)^{*}(v_2), v_1 \rangle &= \langle v_2, ST(v_1) \rangle \\
&= \langle S^{*}(v_2), T(v_1) \rangle \\
&= \langle T^{*}S^{*}(v_2), v_1 \rangle \\
\end{align*}
By uniqueness of the adjoint, then, we must have $(ST)^{*} = T^{*}S^{*}$
\end{proof}

\subsection*{Part (iv)}

\begin{proof}
\begin{align*}
\langle (T^{-1})^{*}(v_2), v_1 \rangle &= \langle (T^{-1})^{*}(v_2), T(T^{-1}(v_1)) \rangle \\
&= \langle T^{*}(T^{-1})^{*}(v_2), T^{-1}(v_1) \rangle \\
&= \langle (T^{-1})^{*}T^{*}(T^{-1})^{*}(v_2), v_1 \rangle \\
\end{align*}
By uniqueness of the adjoint, then, we must have $(T^{-1})^{*}T^{*} = I$, i.e. $(T^{*})^{-1} = (T^{-1})^{*}$
\end{proof}


\section*{Problem 40}

\subsection*{Part (i)}

\begin{proof}
Consider three matrices $A, B, C \in M_n(\mathbb{F})$. Then, we have that:
\begin{align*}
\langle C, AB \rangle &= \langle A^{*}C, B \rangle \\
tr(C^{H}AB) &= tr((A^{*}C)^{H}B) \\
&= tr(C^{H}(A^{*})^{H}B) \\
\end{align*}
Note that setting $A^{*} = A^{H}$ solves the above equation since $(A^{H})^{H} = A$. Thus, by uniqueness of the adjoint we must have $A^{*} = A^{H}$
\end{proof}

\subsection*{Part (ii)}

\begin{proof}
Consider three matrices $A_1, A_2, A_3 \in M_n(\mathbb{F})$. Then, we have that:
\begin{align*}
\langle A_2, A_{3}A_{1} \rangle &= tr((A_{2})^{H}A_{3}A_{1}) \\
&= tr(A_{1}(A_{2})^{H}A_{3}) \\
&= tr((A_{2}(A_{1})^{H})^{H}A_{3}) \\
&= tr((A_{2}A_{1}^{*})^{H}A_{3}) \\
&= \langle A_{2}A_1^{*}, A_{3} \rangle
\end{align*}
Note that setting $A^{*} = A^{H}$ solves the above equation since $(A^{H})^{H} = A$. Thus, by uniqueness of the adjoint we must have $A^{*} = A^{H}$
\end{proof}

\subsection*{Part (iii)}

\begin{proof}
Consider three matrices $A, X, Y \in M_n(\mathbb{F})$. Then, we have that:
\begin{align*}
\langle (T_{A})^{*}(Y), X \rangle &= \langle Y, T_{A}(X) \rangle \\
&= \langle Y, AX - XA \rangle \\
&= \langle Y, AX \rangle - \langle Y, XA \rangle \\
&= tr(Y^{H}AX) - tr(Y^{H}XA) \\
&= \langle A^{H}Y, X \rangle - \langle YA^{H}, X \rangle \\
&= \langle A^{H}Y - YA^{H}, X \rangle \\
&= \langle A^{*}Y - YA^{*}, X \rangle \\
&= \langle T_{A^{*}}(Y), X \rangle \\
\end{align*}
Thus, by uniqueness of the adjoint we must have $(T_{A})^{*} = T_{A^{*}}$
\end{proof}

\section*{Problem 44}

\begin{proof}
By the fundamental subspaces theorem,
$$\mathcal{R}(A)^{\perp} = \mathcal{N}(A^{*}) = \mathcal{N}(A^{H})$$
Note that if $Ax = b$ has no solution $x \in \mathbb{F}^n$, then $b \notin \mathcal{R}(A)$. In this case, we must show that $\exists \ y \in \mathcal{N}(A^{H}) = \mathcal{R}(A)^{\perp}$ s.t. $\langle y, b \rangle \ne 0$. \\
\\
Suppose not. Then $\forall y \in \mathcal{R}(A)^{\perp}$, we must have $\langle y, b \rangle = 0$. Then by definition, $b \in (\mathcal{R}(A)^{\perp})^{\perp} = \mathcal{R}(A)$ But $b \notin \mathcal{R}(A)$, so we have reached a contradiction. \\
\\
Thus, if $Ax = b$ has no solution $x \in \mathbb{F}^n$, then $\exists \ y \in \mathcal{N}(A^{H})$ s.t. $\langle y, b \rangle \ne 0$.
For the case in which $Ax = b$ has some solution $x \in \mathbb{F}^n$, then we know that $b \in \mathcal{R}(A)$. So, $\forall y \in \mathcal{N}(A^{H}) = \mathcal{R}(A)^{\perp}$, we must have $\langle y, b \rangle = 0$ by the definition of $\mathcal{R}(A)^{\perp}$. So, we have established the Fredholm alternative.
\end{proof}

\section*{Problem 45}

\begin{proof}
By definition,
$$Sym_{n}(\mathbb{R}) = \{ A \in M_{n}(\mathbb{R}) \ \vert \ A^{T} = A \}$$
$$Skew_{n}(\mathbb{R}) = \{ A \in M_{n}(\mathbb{R}) \ \vert \ A^{T} = -A \}$$
We must show that:
$$Skew_{n}(\mathbb{R}) = Sym_{n}(\mathbb{R})^{\perp} = \{ A \in M_{n}(\mathbb{R}) \ \vert \ \langle B, A \rangle = 0 \ \forall \ B \in Sym_{n}(\mathbb{R}) \}$$
Consider $A \in Sym_{n}(\mathbb{R})^{\perp}$, $B \in Sym_{n}(\mathbb{R})$ defined as:
$$A \coloneqq \begin{bmatrix} \alpha_{1, 1} & \alpha_{1, 2} & \dots & \alpha_{1, n} \\ \alpha_{2, 1} & \alpha_{2, 2} & \dots & \alpha_{2, n} \\ \vdots & \vdots & \ddots & \vdots \\ \alpha_{n, 1} & \alpha_{n, 2} & \dots & \alpha_{n, n} \end{bmatrix}$$
$$B \coloneqq \begin{bmatrix} \beta_{1, 1} & \beta_{1, 2} & \dots & \beta_{1, n} \\ \beta_{2, 1} & \beta_{2, 2} & \dots & \beta_{2, n} \\ \vdots & \vdots & \ddots & \vdots \\ \beta_{n, 1} & \beta_{n, 2} & \dots & \beta_{n, n} \end{bmatrix}$$
Then, we have that:
$$\langle B, A \rangle = tr(B^{T}A) =  tr(BA) = 0$$
Note that:
$$tr(BA) = \sum\limits_{j = i} (\beta_{i, j}\alpha_{i, j}) + \sum\limits_{j > i} (\beta_{i, j}\alpha_{j, i} + \beta_{j, i}\alpha_{i, j}) \ \text{for $1 \leq j \leq n$, $1 \leq i \leq n$}$$
In order for $tr(BA) = 0 \ \forall \ B \in Sym_{n}(\mathbb{R})$, we must have that for $1 \leq j \leq n$, $1 \leq i \leq n$, $j = i$, $(\beta_{i, j}\alpha_{i, j}) = 0$, and for $1 \leq j \leq n$, $1 \leq i \leq n$, $j > i$, $(\beta_{i, j}\alpha_{j, i} + \beta_{j, i}\alpha_{i, j}) = 0$ for all choices of $B \in Sym_{n}(\mathbb{R})$ with $B$ defined as above. This is easily proved by contradiction. \\
\\
It follows that for $1 \leq j \leq n$, $1 \leq i \leq n$, $j = i$, $\alpha_{i, j} = 0$, and for $1 \leq j \leq n$, $1 \leq i \leq n$, $j > i$, $\alpha_{j, i} = -\alpha_{i, j}$, since $\beta_{i, j} = \beta_{j, i}$. If $A$ satisfies these conditions, then $A \in Skew_{n}(\mathbb{R})$, and if  $A \in Skew_{n}(\mathbb{R})$, then $A$ satisfies these conditions. Thus, we have that $$Skew_{n}(\mathbb{R}) = Sym_{n}(\mathbb{R})^{\perp}$$
\end{proof}


\section*{Problem 46}

\subsection*{Part (i)}

\begin{proof}
Clearly, $Ax \in \mathcal{R}(A)$, by definition. Suppose $x \in \mathcal{N}(A^{H}A)$. Then $A^{H}(Ax) = 0$, so $Ax \in \mathcal{N}(A^{H})$.
\end{proof}

\subsection*{Part (ii)}

\begin{proof}
Suppose $x \in \mathcal{N}(A^{H}A)$. Then:
\begin{align*}
A^{H}Ax &= 0 \\
x^{H}A^{H}Ax &= 0 \\
(Ax)^{H}Ax &= 0 \\
Ax &= 0 \\
x &\in \mathcal{N}(A) \\
\end{align*}
Suppose $x \in \mathcal{N}(A)$. Then $Ax = 0$, so $A^{H}Ax = 0$, so $x \in \mathcal{N}(A^{H}A)$.
Thus, $\mathcal{N}(A^{H}A) = \mathcal{N}(A)$
\end{proof}

\subsection*{Part (iii)}

\begin{proof}
For an $m \times n$ matrix $A$, by the rank-nullity theorem we have that:
$$n = dim(\mathcal{R}(A)) + dim(\mathcal{N}(A))$$
For an $n \times n$ matrix $A^{H}A$, by the rank-nullity theorem we have that:
$$n = dim(\mathcal{R}(A^{H}A)) + dim(\mathcal{N}(A^{H}A))$$
B part (ii), we have that $\mathcal{N}(A^{H}A) = \mathcal{N}(A)$, and in particular that $dim(\mathcal{N}(A^{H}A)) = dim(\mathcal{N}(A))$. So, we have that $dim(\mathcal{R}(A)) = dim(\mathcal{R}(A^{H}A))$, i.e. $A$ and $A^{H}A$ have the same rank.
\end{proof}

\subsection*{Part (iv)}

\begin{proof}
If $A$ has linearly independent columns, then it has rank $n$. By part (iii), $A^{H}A$ must also have rank $n$. Since $A^{H}A$ is an $n \times n$ matrix then (i.e. it has $n$ columns), it must be nonsingular.
\end{proof}


\section*{Problem 47}

\subsection*{Part (i)}

\begin{proof}
Define:
$$P \coloneqq A(A^{H}A)^{-1}A^{H}$$
Then we have that:
\begin{align*}
P^{2} &=  A(A^{H}A)^{-1}A^{H}A(A^{H}A)^{-1}A^{H} \\
&= A(A^{H}A)^{-1}((A^{H}A)(A^{H}A)^{-1})A^{H} \\
&= A(A^{H}A)^{-1}A^{H} \\
&= P
\end{align*}
\end{proof}

\subsection*{Part (ii)}

\begin{proof}
\begin{align*}
P^{H} &=  (A(A^{H}A)^{-1}A^{H})^{H} \\
&= (A^{H})^{H}((A^{H}A)^{-1})^{H}A^{H} \\
&=  A((A^{H}A)^{H})^{-1}A^{H} \\
&= A(A^{H}A)^{-1}A^{H} \\
&= P \\
\end{align*}
\end{proof}

\subsection*{Part (iii)}

\begin{proof}
Note that $rank(A^{H}) = n$ since row-rank = column-rank. Also, note that $(A^{H}A)^{-1}$ is an $n \times n$ matrix of rank $n$ since by problem 46, $A^{H}A$ has rank $n$, and a matrix has the same rank as its inverse. Then we must have $rank(A(A^{H}A)^{-1}) = n$ since A is an $m \times n$ matrix. We know that $rank(A(A^{H}A)^{-1}A^{H}) \leq min\{rank(A(A^{H}A)^{-1}), rank(A^{H})\}$, with equality holding since $rank(A(A^{H}A)^{-1}) = rank(A^{H}) = n$. Thus, $rank(P) = rank(A(A^{H}A)^{-1}A^{H}) = n$.
\end{proof}


\section*{Problem 48}

\subsection*{Part (i)}

\begin{proof}
\begin{align*}
P(A + B) &= \frac{(A + B) + (A + B)^{T}}{2} \\
&=  \frac{A + B + A^{T} + B^{T}}{2} \\
&= \frac{A + A^{T}}{2} + \frac{B + B^{T}}{2} \\
&= P(A) + P(B)
\end{align*}
\begin{align*}
P(cA) &= \frac{cA + (cA)^{T}}{2} \\
&=  \frac{cA + c(A^{T})}{2} \\
&= c \cdot \frac{A + A^{T}}{2} \\
&= cP(A)
\end{align*}
Thus, P is linear.
\end{proof}

\subsection*{Part (ii)}

\begin{proof}
\begin{align*}
P^{2}(A) &= \frac{\frac{A + A^{T}}{2} + (\frac{A + A^{T}}{2})^{T}}{2} \\
&= \frac{\frac{A + A^{T}}{2} + \frac{A^{T} + A}{2}}{2} \\
&= \frac{A + A^{T}}{2} \\
&= P(A) \\
\end{align*}
\end{proof}

\subsection*{Part (iii)}

\begin{proof}
\begin{align*}
P^{*}(A) &= P^{T}(A) \\
&= (\frac{A + A^{T}}{2})^{T} \\
&= \frac{A^{T} + A}{2} \\
&= P(A) \\
\end{align*}
\end{proof}

\subsection*{Part (iv)}

\begin{proof}
If $P(A) = 0$, then $\frac{A + A^{T}}{2} = 0$, so $A + A^{T} = 0$, i.e. $A^{T} = -A$. Thus, $A \in Skew_{n}(\mathbb{R})$. Also, if $A \in Skew_{n}(\mathbb{R})$, it is clear that $P(A) = 0$. Thus $\mathcal{N}(P) = Skew_{n}(\mathbb{R})$
\end{proof}

\subsection*{Part (v)}

\begin{proof}
\begin{align*}
\mathcal{R}(P)^{\perp} &= \mathcal{R}(P^{T})^{\perp} &&\text{by part (iii)} \\
&= \mathcal{N}(P) &&\text{by the fundamental subspaces theorem} \\
&= Skew_{n}(\mathbb{R}) &&\text{by part (iv)} \\
&= Sym_{n}(\mathbb{R})^{\perp} &&\text{by problem 45} \\
\end{align*}
Thus, we have that $\mathcal{R}(P)^{\perp} = Sym_{n}(\mathbb{R})^{\perp}$, so $\mathcal{R}(P)= Sym_{n}(\mathbb{R})$
\end{proof}

\subsection*{Part (vi)}

\begin{proof}
\begin{align*}
\|A - P(A)\|_F &= \sqrt{tr((A - P(A))^{T}(A - P(A)))} \\
&= \sqrt{tr\left(\left(A - \frac{A + A^{T}}{2}\right)^{T}\left(A - \frac{A + A^{T}}{2}\right)\right)} \\
&= \sqrt{tr\left(\left(\frac{A - A^{T}}{2}\right)^{T}\left(\frac{A - A^{T}}{2}\right)\right)} \\
&= \sqrt{tr\left(\frac{A^{T} - A}{2} \cdot \frac{A - A^{T}}{2}\right)} \\
&= \sqrt{tr\left(\frac{A^{T}A - A^{T}A^{T} - AA + AA^{T}}{4}\right)} \\
&= \sqrt{\frac{tr(A^{T}A) + tr(AA^{T}) - tr((AA)^{T}) - tr(AA)}{4}} \\
&= \sqrt{\frac{tr(A^{T}A) + tr(A^{T}A) - tr(AA) - tr(AA)}{4}} \\
&= \sqrt{\frac{2tr(A^{T}A) - 2tr(AA)}{4}} \\
&= \sqrt{\frac{tr(A^{T}A) - tr(A^{2})}{2}} \\
\end{align*}
\end{proof}


\section*{Problem 50}
Define the following (note that b is an $n$-vector):
$$A \coloneqq \begin{bmatrix} x_1^2 & y_1^2 \\ x_2^2 & y_2^2  \\ \vdots & \vdots \\ x_n^2 & y_n^2 \end{bmatrix}$$
$$x \coloneqq \begin{bmatrix} r \\ s \end{bmatrix}$$
$$b \coloneqq \begin{bmatrix} 1 \\ 1 \\ \vdots \\ 1 \end{bmatrix}$$


\end{document}