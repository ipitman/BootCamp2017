\documentclass{article}
\usepackage{amsmath, amsthm, amssymb}
\begin{document}

\section{Problem 3.6:}

\begin{proof}
$B_i \cap B_j = \emptyset$ $\forall i \ne j$, so $(A \cap B_i) \cap (A \cap B_j) = \emptyset$ $\forall i \ne j$, i.e. $\{A \cap B_i\}_{i \in I}$ is a collection of pairwise-disjoint events. Since it is indexed by a finite or countable set $I$, then, we have that:
\begin{align*}
\sum\limits_{i \in I} P(A \cap B_i) &= P(\bigcup\limits_{i \in I} A \cap B_i) \\
&= P(A \cap (\bigcup\limits_{i \in I} B_i)) &&\text{by the distributive property of $\cap$} \\
&= P(A \cap \Omega) &&\text{by assumption} \\
&= P(A) &&\text{since $A \subseteq \Omega$}
\end{align*}
\end{proof}

\section{Problem 3.8:}

\begin{proof}
\begin{align*}
P(\bigcup\limits_{k = 1}^n E_k) &= 1 - P((\bigcup\limits_{k = 1}^n E_k)^C) \\
&= 1 - P(\bigcap\limits_{k = 1}^n E_k^C) &&\text{by De Morgan's law} \\
&= 1 - \prod\limits_{k = 1}^n P(E_k^C) &&\text{since the events are independent} \\
&= 1 - \prod\limits_{k = 1}^n (1 - P(E_k))
\end{align*}
\end{proof}

\section{Problem 3.11:}
Who knows
\section{Problem 3.12:}
\begin{proof}
Consider the mutually exclusive and collectively exhaustive set of events $\{C_1, C_2, C_3\}$, where $C_n$ corresponds to the car being behind the $n$th door.
Also, consider the set of events $\{D_1, D_2, D_3\}$ where $D_n$ corresponds to Monty Hall opening the $n$th door.
Suppose you choose the 1st door. Then, we know that:
\begin{align*}
P(D_3 | C_1) &= \frac{1}{2} &&\text{since Monty Hall is equally likely to choose to open the 1st or the 2nd door} \\
P(D_3 | C_2) &= 1 &&\text{since Monty Hall must choose to open the 3rd door since you have chosen the 1st door} \\
P(D_3 | C_3) &= 0 &&\text{since Monty Hall cannot reveal the car} \\
\end{align*}
Clearly, we have that:
$$P(C_1) = P(C_2) = P(C_3) = \frac{1}{3}$$
So, we have that:
\begin{align*}
P(C_2 | D_3) &= \frac{P(D_3 | C_2) P(C_2)}{P(D_3 | C_1) P(C_1) + P(D_3 | C_2) P(C_2) + P(D_3 | C_3) P(C_3)} \\
&= \frac{1 \cdot \frac{1}{3}}{\frac{1}{2} \cdot \frac{1}{3} + 1 \cdot \frac{1}{3} + 0 \cdot \frac{1}{3}} \\
&= \frac{2}{3} \\
P(C_1 | D_3) &= \frac{P(D_3 | C_1) P(C_1)}{P(D_3 | C_1) P(C_1) + P(D_3 | C_2) P(C_2) + P(D_3 | C_3) P(C_3)} \\
&= \frac{\frac{1}{2} \cdot \frac{1}{3}}{\frac{1}{2} \cdot \frac{1}{3} + 1 \cdot \frac{1}{3} + 0 \cdot \frac{1}{3}} \\
&= \frac{1}{3} \\
\end{align*}
Note that we may replace $D_3$ with $D_2$ in the above equations and arrive at the same result. Since we know that one of $D_3$ and $D_2$ must occur, it follows that the probability of winning is higher if the contestant changes to the other unopened door (the 2nd door) instead of sticking with the original choice (the 1st door) since:
$$P(C_2 | D_3) = P(C_2 | D_2) = \frac{2}{3} > \frac{1}{3} = P(C_1 | D_2) = P(C_1 | D_3)$$
If there were 10 doors and Monty opened 8 with goats after the contestant's first choice, the contestant would have a $\frac{9}{10}$ chance of winning by switching, and a $\frac{1}{10}$ chance of winning by sticking with the original choice. This result can be found by a trivial generalization of the above proof (it is useful in this case to think of events $D_n$ as Monty Hall {\it not} opening the $n$th door).
\end{proof}
\end{document}